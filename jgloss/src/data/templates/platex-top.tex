% Encoding: EUC-JP
% This is a EUC-JP encoded Japanese LaTeX document generated
% by JGloss from %document-filename% on %generation-time%.
% Use platex to process this file.
% Note that the file ruby-annotation.sty which comes with JGloss
% must be installed.
%
% template-author: Michael Koch <tensberg@gmx.net>
% The template file may be freely redistributed unmodified
% or modified.
% $Id$

\documentclass[%font-size%pt]{jarticle}
\usepackage[overlap,CJK]{ruby-annotation}
\usepackage{pslatex}
\renewcommand{\thefootnote}{}

\begin{document}

\pagestyle{myheadings}
\markright
\newlength{\ww}
\settowidth{\ww}{%longest-word%--}
\newlength{\rw}
\settowidth{\rw}{%longest-reading%--}

\newlength{\tw}
\setlength{\tw}{\textwidth}
\addtolength{\tw}{-1.0\ww}
\addtolength{\tw}{-1.0\rw}
% I don't know where the -20pt come from, but they have to be there for
% the layout to work.
\addtolength{\tw}{-20pt}
\newcommand{\fn}[3]{\footnotetext{\makebox[\ww][l]{#1} \makebox[\rw][l]{#2} \parbox[t]{\tw}{#3}}}

%begin text
%ruby \\ruby
%translation \\fn
%end text

\end{document}
